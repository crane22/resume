% !TEX program = xelatex
\documentclass{resume}
\usepackage{hyperref}
% \usepackage{cite}

\begin{document}
\pagenumbering{gobble}          % suppress displaying page number


\name{GUO Yaowei}
\basicInfo{
    \phone{(+86) 186-0096-3933}
    \  \textperiodcentered \:
    \email{i@crane.moe}
    \  \textperiodcentered \:
    \github[@crane22]{https://github.com/crane22}
    % \  \textperiodcentered \:
    % \linkedin[billryan8]{https://www.linkedin.com/in/billryan8}
}


\section{\faGraduationCap \: Education}
\datedsubsection{\textbf{Beijing University of Posts and Telecommunications}, B. Eng.}{2019.09 -- 2023.06 (Expected)}
\datedsubsection{Major in \textbf{Information Engineering}, \textit{School of Information and Commuication Eng.}}{Beijing, China}
\begin{itemize}
    \item \textbf{GPA}: 3.2/4.0
    \item \textbf{Award}: Third-class Scholarship, Outstanding Class Cadre
    \item \textbf{Key Courses}: Pattern Recognition and its Applications(93), Network Theory Basics(91), Web Search(94), Big Data Applications and Experiments(90), etc
\end{itemize}

\datedsubsection{\textbf{Beijing University of Posts and Telecommunications}, Minor}{2020.01 -- 2023.06 (Expected)}
\datedsubsection{Research Topic: \textbf{Smart Traffic}, \textit{School of YE Peida Innovation and Entrepreneurship}}{Beijing, China}
\begin{itemize}
    \item \textbf{Lab}: YE Peida Innovation Lab, supervised by \href{https://teacher.bupt.edu.cn/dai/zh_CN/index.htm}{Prof.DAI Zhitao}
\end{itemize}


\section{\faSearch \: Research Experience}
\datedsubsection{\textbf{Covid-19 Data Analysis and Prediction}}{2022.05 -- 2022.06}
\datedsubsection{An Individual Project}{\textit{Full-stack Development}}
Fetch Covid-19 data from China's National Health Commission, and predict data with accuracy of $\sim$85\%.
\begin{itemize}
    \item \textbf{Predictive Modeling}: Predict Sequential Data with LSTM and GRU
    \item \textbf{AutoML Procedure}: Deploy AutoML framework  (\href{https://github.com/microsoft/nni}{microsoft/nni}) and reduce tuning time by $\sim$30\%.
    \item \github[crane22/Covid-19\_ChinaMainland\_Prediction\_LSTM-GRU]{https://github.com/crane22/Covid-19_ChinaMainland_Prediction_LSTM-GRU}
\end{itemize}

\datedsubsection{\textbf{Parking Violation Capture Drone}}{2021.08 -- 2022.07}
\datedsubsection{A \textit{YE Peida Innovation Lab} Project}{\textit{Algorithm (Python)}}
Detect parking violation on a drone. \textbf{The project won the \nth{2} prize in the "Internet+" Competition.}
\begin{itemize}
    \item \textbf{System Design and Development}: Design and develop the system in three divided modules:
    \item A Semantic Segmentation method based on \href{https://arxiv.org/abs/2205.08534v3}{ViT} to differentiate vehicle from the environment
    \item An \href{https://github.com/open-mmlab/mmflow}{Optical Flow motion analyzer} based on \href{https://arxiv.org/abs/2104.02409}{GMA} to detect whether the vehicle is moving or parked
    \item A lane detector based on \href{https://arxiv.org/abs/1804.02767}{YOLOv3} to judge whether the vehicle parked at a legitimate place
\end{itemize}

\datedsubsection{\textbf{Human Body Detection System}}{2021.06 -- 2022.06}
\datedsubsection{A Lab Project}{\textit{Leader and Algorithm}}
An end-to-end solution of human body detection on Depth Cameras.
\begin{itemize}
    \item \textbf{Dataset Building} and \textbf{Data Collation}: Collect data and build a dataset on two Depth Cameras.
    \item \textbf{Solution Delivery}: Design an end-to-end solution based on models like YOLOv3, SSD and Faster-RCNN.
\end{itemize}

\datedsubsection{\textbf{Face Recognition Door Guard}}{2019.09 -- 2020.01}
\datedsubsection{An Individual Project}{\textit{Hardware and Python}}
A simple Face Recognition Door Guard runs on a RaspberryPi and an Arduino Uno.
\begin{itemize}
    \item \textbf{System Design}: Utilized Haar Cascade classifier to implement a real-time face recognition using OpenCV. 
    \item \github[crane22/FaceRecognitionDoorGuard\_Prototype]{https://github.com/crane22/FaceRecognitionDoorGuard_Prototype}
\end{itemize}
% % Reference Test
% \datedsubsection{\textbf{Paper Title\cite{zaharia2012resilient}}}{May. 2015}
% An xxx optimized for xxx\cite{verma2015large}
% \begin{itemize}
%     \item main contribution
% \end{itemize}


% \section{\faUsers \: Work Experience}
% \datedsubsection{\textbf{FLAG Inc.} California, America}{2012 -- Present}
% \role{Summer Intern}{Manager: xxx}
% Brief introduction: xxx.
% \begin{itemize}
%     \item Implemented xxx feature
%     \item Optimized xxx 5\%
%     \item xxx
% \end{itemize}


% \section{\faStarO \: Open-Source Contributions}
% \datedsubsection{\textbf{\LaTeX\ résumé template}}{May. 2015 -- Present}
% \role{\LaTeX, Maintainer}{Individual Projects}
% An elegant \LaTeX\ résumé template, https://github.com/billryan/resume
% \begin{itemize}
%     \item Easy to be further customized or extended
%     \item Full support for unicode characters (e.g. CJK) with \XeLaTeX\
%     \item FontAwesome 4.5.0 support
% \end{itemize}


\section{\faCogs \: Skills}
\begin{itemize}[parsep=0.5ex]
    \item \textbf{Programming Languages}: Experienced in \textbf{Python} and \textbf{C/C++}, comfortable with \textbf{Java} and \textbf{Rust}, but not limited to any specific language.
    \item \textbf{Machine Learning}: Experienced in frameworks like \textbf{Pytorch} and \textbf{NumPy}, and AutoML tools like \textbf{NNI}.
    \item \textbf{Developing Tools and Platforms}: Experienced in \textbf{Linux-based programming}, and tools like \textbf{Git}.
    \item \textbf{Human Languages}: \textbf{Mandarin} - Native speaker,  \textbf{English} - Fluent (CET-6 scored 566 points)
\end{itemize}


% \section{\faHeartO \: Honors and Awards}
% \datedline{\textit{\nth{1} Prize}, Award on xxx }{Jun. 2013}
% \datedline{Other awards}{2015}


% \section{\faInfo \: Miscellaneous}
% \begin{itemize}[parsep=0.5ex]
%     \item 
% \end{itemize}


% % Reference
% \newpage
% \bibliographystyle{IEEETran}
% \bibliography{mycite}
\end{document}
