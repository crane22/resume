% !TEX TS-program = xelatex
% !TEX encoding = UTF-8 Unicode
% !Mode:: "TeX:UTF-8"
\documentclass{resume}
\usepackage{zh_CN-Adobefonts_external} % Simplified Chinese Support using external fonts (./fonts/zh_CN-Adobe/)
% \usepackage{zh_CN-Adobefonts_internal} % Simplified Chinese Support using system fonts
% \usepackage{NotoSansSC_external}
% \usepackage{NotoSerifCJKsc_external}
\usepackage{linespacing_fix} % disable extra space before next section
\usepackage{hyperref}
% \usepackage{cite}

\begin{document}
\pagenumbering{gobble} % suppress displaying page number


\name{郭曜玮}
\basicInfo{
  \phone{(+86) 186-0096-3933}
  \  \textperiodcentered \:
  \email{i@crane.moe}
  \  \textperiodcentered \:
  \github[@crane22]{https://github.com/crane22}
  %\linkedin[billryan8]{https://www.linkedin.com/in/billryan8}
}


\section{\faGraduationCap \: 教育背景}
\datedsubsection{\textbf{北京邮电大学},学士学位}{2019.09 -- (预计)2023.06}
\datedsubsection{\textbf{信息工程}专业,\textit{信息与通信工程学院}}{北京}
\begin{itemize}
    \item \textbf{GPA}: 3.2/4.0
    \item \textbf{奖项}: 三等奖学金,校先进班干部
    \item \textbf{主要课程}: 模式识别与应用(93),网络理论基础(91),Web搜索(94),大数据应用实践(90),等
\end{itemize}

\datedsubsection{\textbf{北京邮电大学},辅修专业}{2020.01 -- (预计)2023.06}
\datedsubsection{\textbf{智慧交通}研究方向,\textit{叶培大创新创业学院}}{北京}
\begin{itemize}
    \item \textbf{实验室}: 叶培大创新创业基地,指导老师\href{https://teacher.bupt.edu.cn/dai/zh_CN/index.htm}{戴志涛教授}
\end{itemize}


\section{\faSearch \: 项目经历}
\datedsubsection{\textbf{疫情数据分析预测系统}}{2022.05 -- 2022.06}
\datedsubsection{个人项目}{\textit{全栈开发}}
从国家卫健委官网爬取疫情数据,并通过机器学习算法进行预测。
\begin{itemize}
  \item 通过 LSTM 和 GRU 预测顺序数据
  \item 使用 \href{https://github.com/microsoft/nni}{microsoft/nni} 调整超参数、搜索网络架构
  \item \github[crane22/Covid-19\_ChinaMainland\_Prediction\_LSTM-GRU]{https://github.com/crane22/Covid-19_ChinaMainland_Prediction_LSTM-GRU}
\end{itemize}

\datedsubsection{\textbf{违法停车检测无人机}}{2021.08 -- 2022.07}
\datedsubsection{\textit{叶培大创新创业学院} 项目}{\textit{算法 (Python)}}
从无人机获取实时视频流,并分析检测是否存在违法停车现象。\textbf{本项目获得“互联网+”比赛北京赛区二等奖。} 算法部分主要由三部分组成:
\begin{itemize}
  \item 使用基于 \href{https://arxiv.org/abs/2205.08534v3}{ViT} 的语义分割方法,将车辆和地面进行区分
  \item 通过基于 \href{https://arxiv.org/abs/2104.02409}{GMA} 的\href{https://github.com/open-mmlab/mmflow}{光流运动检测器},检测车辆的运动或静止
  \item 基于 \href{https://arxiv.org/abs/1804.02767}{YOLOv3} 对车道线进行检测,判断车辆是否停在合法的区域
\end{itemize}

\datedsubsection{\textbf{基于深度相机的人体检测系统}}{2021.06 -- 2022.06}
\datedsubsection{实验室项目}{\textit{算法与主要负责人}}
通过深度相机采集数据后,实时分析检测的端到端人体检测系统。
\begin{itemize}
  \item 通过两个深度摄像头收集数据,建立了所需的数据集
  \item 在数据集上测试了诸如 \href{https://arxiv.org/abs/1804.02767}{YOLOv3}、\href{https://arxiv.org/abs/1512.02325v5}{SSD}、和 \href{https://arxiv.org/abs/1506.01497v3}{Faster-RCNN}等目标检测算法
\end{itemize}

\datedsubsection{\textbf{人脸识别门禁}}{2019.09 -- 2020.01}
\datedsubsection{小组项目}{\textit{硬件部分与Python}}
使用树莓派搭建的一个人脸识别门禁系统。
\begin{itemize}
  \item 基于OpenCV库的实时人脸识别检测 
  \item \github[crane22/FaceRecognitionDoorGuard\_Prototype]{https://github.com/crane22/FaceRecognitionDoorGuard_Prototype}
\end{itemize}
% Reference Test
%\datedsubsection{\textbf{Paper Title\cite{zaharia2012resilient}}}{May. 2015}
%An xxx optimized for xxx\cite{verma2015large}
%\begin{itemize}
%  \item main contribution
%\end{itemize}


% \section{\faUsers \: 工作经历}
% \datedsubsection{\textbf{FLAG Inc.} California, America}{2012 -- Present}
% \role{Summer Intern}{Manager: xxx}
% Brief introduction: xxx.
% \begin{itemize}
%   \item Implemented xxx feature
%   \item Optimized xxx 5\%
%   \item xxx
% \end{itemize}


% \section{\faStarO \: 开源贡献}
% \datedsubsection{\textbf{\LaTeX\ résumé template}}{May. 2015 -- Present}
% \role{\LaTeX, Maintainer}{Individual Projects}
% An elegant \LaTeX\ résumé template, https://github.com/billryan/resume
% \begin{itemize}
%   \item Easy to be further customized or extended
%   \item Full support for unicode characters (e.g. CJK) with \XeLaTeX\
%   \item FontAwesome 4.5.0 support
% \end{itemize}


\section{\faCogs\ 技术技能}
\begin{itemize}[parsep=0.5ex]
  \item \textbf{编程语言}: 熟练掌握 \textbf{Python} 和 \textbf{C/C++},略懂 \textbf{Java} 和 \textbf{Rust},但不会受制于语言。
  \item \textbf{开发工具和平台}: 熟悉 \textbf{Linux 下}的工作流,也对诸如\textbf{Git} 和 \textbf{docker}等工具相对熟悉。
  \item \textbf{语言}: \textbf{英语} - 流利 (六级/CET-6 566 分)
\end{itemize}


% \section{\faHeartO\ Honors and Awards}
% \datedline{\textit{\nth{1} Prize}, Award on xxx }{Jun. 2013}
% \datedline{Other awards}{2015}


% \section{\faInfo\ Miscellaneous}
% \begin{itemize}[parsep=0.5ex]
%   \item 
% \end{itemize}


%% Reference
%\newpage
%\bibliographystyle{IEEETran}
%\bibliography{mycite}
\end{document}
